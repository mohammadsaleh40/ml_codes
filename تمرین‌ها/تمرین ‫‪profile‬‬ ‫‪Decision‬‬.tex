\documentclass[12pt]{article}

\usepackage{amsmath}
\usepackage{graphicx}
\usepackage{xepersian}
\settextfont{Yas}
\setdigitfont{A Iranian Sans}

\title{تمرین \lr{Profile Decision}}
\author{محمد صالح علی اکبری}
\date{تاریخ تحویل: 31 اردیبهشت}

\begin{document}
	
	\maketitle
	
	\section*{سؤال ۱}
	یک جدول مشخصات تصمیم‌گیری \lr{Decision Profile} برای چهار طبقه‌بند و سه کلاس رسم کرده و مقادیر دلخواه را در آن وارد کنید.
	
	\subsection*{پاسخ}
	مثال جدول \lr{Decision Profile} برای \(L=4\) طبقه‌بند و \(C=3\) کلاس به‌صورت زیر است:
	
	\begin{table}[h!]
		\centering
		\caption{جدول \lr{Decision Profile} برای یک نمونه \(x\)}
		\begin{tabular}{lccc}
			\hline
			\textbf{طبقه‌بند / کلاس} & \textbf{کلاس 1} & \textbf{کلاس 2} & \textbf{کلاس 3} \\
			\hline
			طبقه‌بند 1 & 0.30 & 0.45 & 0.25 \\
			طبقه‌بند 2 & 0.20 & 0.50 & 0.30 \\
			طبقه‌بند 3 & 0.40 & 0.35 & 0.25 \\
			طبقه‌بند 4 & 0.25 & 0.30 & 0.45 \\
			\hline
		\end{tabular}
	\end{table}
	
	در این جدول، هر سطر خروجی یک طبقه‌بند را نشان می‌دهد. مقدار سلول \((i,j)\) یعنی \(d_{i,j}(x)\)، بیانگر میزان حمایت طبقه‌بند \(i\) از کلاس \(j\) برای نمونه \(x\) است.
	
	\section*{سؤال ۲}
	عملگرهای مختلف برای تابع ترکیبی \lr{Combination Function} را با فرمول بیان کنید.
	
	\subsection*{پاسخ}
	برای محاسبه حمایت کلی \(m_j(x)\) از کلاس \(\omega_j\)، عملگر ترکیبی \(F\) را بر ستون \(j\)، یعنی \([d_{1,j}(x),\dots,d_{L,j}(x)]\)، اعمال می‌کنیم:
	
	\[
	m_j(x)=F\bigl(d_{1,j}(x),\dots,d_{L,j}(x)\bigr)
	\]
	
	نمونه‌هایی از توابع \(F\):
	
	\begin{description}
		\item[میانگین ساده \lr{(Simple Mean)}:]
		\[
		m_j(x) = \frac{1}{L} \sum_{i=1}^{L} d_{i,j}(x)
		\]
		
		\item[وزن‌دار \lr{(Weighted Sum)}:]
		\[
		m_j(x) = \sum_{i=1}^{L} w_i\,d_{i,j}(x), \quad w_i \ge 0,\ \sum_{i=1}^{L} w_i = 1
		\]
		
		\item[حداکثر \lr{(Maximum)}:]
		\[
		m_j(x) = \max_{i} \{ d_{i,j}(x) \}
		\]
		
		\item[حداقل \lr{(Minimum)}:]
		\[
		m_j(x) = \min_{i} \{ d_{i,j}(x) \}
		\]
		
		\item[میانه \lr{(Median)}:]
		\[
		m_j(x) = \mathrm{median} \bigl\{ d_{1,j}(x), \dots, d_{L,j}(x) \bigr\}
		\]
		
		\item[میانگین هرس‌شده \lr{(Trimmed Mean)}:] ابتدا مقادیر را مرتب کرده و درصد \(K\) از ابتدا و انتها را حذف می‌کنیم. سپس میانگین مقادیر باقی‌مانده را محاسبه می‌کنیم.
		
		\item[ضرب \lr{(Product)}:]
		\[
		m_j(x) = \prod_{i=1}^{L} d_{i,j}(x)
		\]
		
		\item[میانگین عمومی \lr{(Generalized Mean)}:]
		\[
		m_j(x; a) = \left( \frac{1}{L} \sum_{i=1}^{L} [d_{i,j}(x)]^a \right)^{1/a}
		\]
		موارد خاص:
		\begin{itemize}
			\item \(a \to +\infty\): حداکثر \lr{(Max)}
			\item \(a = 1\): میانگین حسابی \lr{(Arithmetic Mean)}
			\item \(a = 0\): میانگین هندسی \lr{(Geometric Mean)}
			\item \(a = -1\): میانگین هارمونیک \lr{(Harmonic Mean)}
			\item \(a \to -\infty\): حداقل \lr{(Min)}
		\end{itemize}
	\end{description}
	
\end{document}
