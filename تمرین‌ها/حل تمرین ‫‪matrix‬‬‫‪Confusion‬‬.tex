\documentclass[12pt]{article}

\usepackage{amssymb}
\usepackage{amsmath}
\usepackage{graphicx}
\usepackage{xepersian}
\settextfont{Yas}
\setdigitfont{A Iranian Sans}

\title{تمرین \lr{Confusion Matrix}}
\author{محمد صالح علی‌اکبری}
\date{تاریخ تحویل: ۲ خرداد}

\begin{document}
	
	\maketitle
	
	\section*{سؤال ۱}
	‫مشتقات‬ ‫فرمول‬ \lr{Derivations‬‬ ‫‪Formul}‬ ‫در‬ ‫ترکیب‬ ‫بیز‬ ‫ساده‬ ‫را‬ ‫بیان‬ ‫کنید‬.
	
	\subsection*{پاسخ}
	
	با استفاده از قانون بیز برای دسته‌بندی داریم:
	\begin{align*}
		P(c\mid \mathbf{x}) &= \frac{P(\mathbf{x}\mid c)\,P(c)}{P(\mathbf{x})} \\
		&= \frac{P(x_1, x_2, \dots, x_n\mid c)\,P(c)}{\sum_{c'}P(x_1, x_2, \dots, x_n\mid c')\,P(c')}.
	\end{align*}
	
	اگر ویژگی‌ها شرطی مستقل فرض شوند (مدل بیز ساده)، آنگاه:
	\[
	P(x_1, \dots, x_n\mid c) = \prod_{i=1}^n P(x_i\mid c)
	\]
	و در نتیجه مدل نایو بیز خواهیم داشت:
	\[
	P(c\mid \mathbf{x}) = \frac{P(c)\prod_{i=1}^n P(x_i\mid c)}{\sum_{c'}P(c')\prod_{i=1}^n P(x_i\mid c')}.
	\]
	
	\section*{سؤال ۲}
	محاسبه \lr{Support} برای کلاس‌های $w_1$ و $w_2$ را با استفاده از جدول زیر انجام دهید:
	
	\begin{table}[h!]
		\centering
		\caption{ماتریس درهم‌ریختگی}
		\begin{tabular}{c|cc|c}
			واقعی $\backslash$ پیش‌بینی & $w_1$ & $w_2$ & جمع ردیف \\
			\hline
			$w_1$ & 60 & 10 & 70 \\
			$w_2$ & 15 & 55 & 70 \\
			\hline
			جمع ستون & 75 & 65 & 140 \\
		\end{tabular}
	\end{table}
	
	\subsection*{پاسخ}
	
	تعداد کل نمونه‌ها برابر است با:
	\[
	N = 60 + 10 + 15 + 55 = 140
	\]
	
	\begin{itemize}
		\item \lr{Support}($w_1$) = نسبت نمونه‌های واقعی کلاس $w_1$: \[ \frac{60 + 10}{140} = \frac{70}{140} = 0.50 \]
		\item \lr{Support}($w_2$) = نسبت نمونه‌های واقعی کلاس $w_2$: \[ \frac{15 + 55}{140} = \frac{70}{140} = 0.50 \]
	\end{itemize}
	
\end{document}
