\documentclass[12pt]{article}

\usepackage{amssymb}
\usepackage{amsmath} % بسته ضروری برای معادلات
\usepackage{graphicx}
\usepackage{xepersian}
\settextfont{Yas}
\setdigitfont{A Iranian Sans}

\title{تمرین معنا و ریاضیات ریسک}
\author{محمد صالح علی اکبری}
\date{تاریخ تحویل: 31 اردیبهشت}

\begin{document}
	
	\maketitle
	
	\section*{سؤال ۱}
	ریسک نرمال بیشتر با تابع زیان متناظر است یا ریسک تجربی؟ به صورت مفهومی با دلیل توضیح دهید.
	
	\subsection*{پاسخ}
	ریسک نرمال (یا ریسک حقیقی) تعریف‌ شده به صورت توقع (امید ریاضی) تابع زیان روی توزیع داده‌های واقعی است:
	
	\[
	R( f ) = \mathbb{E}_{(x,y)\sim P}[L\bigl(f(x),y\bigr)]
	\]
	
	بنابراین ریسک نرمال دقیقاً انتگرال (یا امید) تابع زیان را با توزیع واقعی در نظر می‌گیرد و به تابع زیان متناظر وابسته است. در مقابل، ریسک تجربی تنها میانگین تابع زیان بر روی نمونه‌های آموزشی است:
	
	\[
	R_{\text{emp}}(f)=\frac{1}{n}\sum_{i=1}^{n}L\bigl(f(x_i),y_i\bigr)
	\]
	
	چون ریسک نرمال به طور مستقیم امید تابع زیان را محاسبه می‌کند، بیشتر با تابع زیان متناظر \lr{(Loss Function)} در ارتباط است؛ اما ریسک تجربی تنها تقریب نمونه‌ای آن است.
	
	\section*{سؤال ۲}
	با توجه به داده‌های زیر ریسک حقیقی، ریسک تجربی و ریسک ساختاری را محاسبه کنید و نتایج را تحلیل کنید.
	
	یک مدل یادگیری ماشین برای یک مسئله رگرسیون آموزش داده شده است. جدول زیر شامل داده‌های واقعی و پیش‌بینی مدل روی ۴ نمونه است: \\
	
	\begin{table}[h!]
		\centering
		\begin{tabular}{lcc}
			\hline
			\textbf{نمونه} & \textbf{مقدار واقعی $y_i$} & \textbf{پیش‌بینی مدل $f(x_i)$} \\
			\hline
			1 & $3.0$ & $2.5$ \\
			2 & $2.0$ & $2.2$ \\
			3 & $4.0$ & $3.0$ \\
			4 & $1.0$ & $1.5$ \\
			\hline
		\end{tabular}
	\end{table}
	\\
	
	برای مدل فوق تابع زیان برابر \lr{MSE} است:
	
	\[
	L\bigl(f(x_i),y_i\bigr) = \bigl(f(x_i)-y_i\bigr)^2
	\]
	
	میزان پیچیدگی مدل با نرم وزن‌ها برابر \(\Omega(f)=4\) و ضریب منظم‌سازی برابر \(\lambda=0.3\) است.
	
	\subsection*{پاسخ}
	\begin{itemize}
		\item \textbf{\lr{Empirical Risk} (ریسک تجربی)}
		
		\[
		R_{\text{emp}}(f)=\frac{1}{4}\sum_{i=1}^{4}(f(x_i)-y_i)^2
		=\frac{1}{4}\bigl[(2.5-3.0)^2+(2.2-2.0)^2+(3.0-4.0)^2+(1.5-1.0)^2\bigr]
		\]
		\[
		=\frac{1}{4}(0.25+0.04+1.00+0.25)=\frac{1.54}{4}=0.385
		\]
		
		\item \textbf{\lr{True Risk} (ریسک حقیقی)}
		
		\[
		R(f)=\mathbb{E}[L(f(x),y)]
		\]
		
		بدون دانستن توزیع کامل \((x,y)\)، نمی‌توان مقدار دقیق ریسک حقیقی را محاسبه کرد. اما اگر این ۴ نمونه را نماینده توزیع فرض کنیم، تقریب آن برابر ریسک تجربی خواهد بود: \(R(f)\approx 0.385\)
		
		\item \textbf{\lr{Structural Risk} (ریسک ساختاری)}
		
		\[
		R_{\text{struct}}(f)=R_{\text{emp}}(f)+\lambda\,\Omega(f)
		=0.385 + 0.3\times 4 =0.385 + 1.2 = 1.585
		\]
		
		\item \textbf{تحلیل نتایج:}
		\begin{itemize}
			\item ریسک تجربی \((0.385)\) نشان می‌دهد مدل روی داده‌های آموزشی خطای متوسط مربعی نسبتاً پایین دارد.
			\item چون توزیع واقعی نامشخص است، ریسک حقیقی دقیقاً مشخص نیست؛ ولی تقریب آن مشابه ریسک تجربی خواهد بود.
			\item ریسک ساختاری بزرگ‌تر \((1.585)\) است؛ زیرا ضریب منظم‌سازی (penalty) را اضافه می‌کند. این مقدار بالا نشان می‌دهد پیچیدگی مدل (وزن‌ها) تأثیر قابل‌توجهی بر کل ریسک دارد.
			
			
		\end{itemize}
	\end{itemize}
	
\end{document}
