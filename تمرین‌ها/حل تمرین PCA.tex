\documentclass[12pt]{article}

\usepackage{amssymb}
\usepackage{amsmath}
\usepackage{graphicx}
\usepackage{xepersian}
\settextfont{Yas}
\setdigitfont{A Iranian Sans}

\title{تمرین PCA}
\author{محمد صالح علی اکبری}
\date{تاریخ تحویل: ۲ خرداد}

\begin{document}
	
	\maketitle
	
	\section*{سؤال ۱}
	مراحل محاسبات \lr{PCA} را با روابط هر مرحله تشریح کنید.
	
	\subsection*{پاسخ}
	مراحل اصلی تحلیل مؤلفه‌های اصلی (\lr{PCA}) به شرح زیر است:
	
	\begin{enumerate}
		\item \textbf{مرکزسازی داده‌ها:} از هر ویژگی میانگین آن را کم می‌کنیم تا داده‌ها حول مبدأ قرار گیرند:
		\[
		x_i^{\text{centered}} = x_i - \mu
		\]
		که در آن $\mu$ میانگین هر ویژگی است.
		
		\item \textbf{محاسبه ماتریس کوواریانس:} با استفاده از داده‌های مرکزسازی‌شده، ماتریس کوواریانس را محاسبه می‌کنیم:
		\[
		\Sigma = \frac{1}{n} X^T X
		\]
		که در آن $X$ ماتریس داده‌های مرکزسازی‌شده و $n$ تعداد نمونه‌ها است.
		
		\item \textbf{محاسبه مقادیر و بردارهای ویژه:} مقادیر ویژه و بردارهای ویژه ماتریس کوواریانس را محاسبه می‌کنیم:
		\[
		\Sigma \phi_i = \lambda_i \phi_i
		\]
		که در آن $\lambda_i$ مقدار ویژه و $\phi_i$ بردار ویژه متناظر است.
		
		\item \textbf{مرتب‌سازی بردارهای ویژه:} بردارهای ویژه را بر اساس مقادیر ویژه متناظر به‌صورت نزولی مرتب می‌کنیم.
		
		\item \textbf{انتخاب مؤلفه‌های اصلی:} $m$ بردار ویژه اول را انتخاب می‌کنیم تا فضای ویژگی کاهش‌یافته را تشکیل دهند.
		
		\item \textbf{پروژه‌سازی داده‌ها:} داده‌های اصلی را روی فضای جدید پروژه می‌کنیم:
		\[
		\hat{x} = \sum_{i=1}^{m} y_i \phi_i
		\]
		که در آن $y_i = \phi_i^T x$ است.
	\end{enumerate}
	
	\section*{سؤال ۲}
	\lr{projected\_data} را تعریف کرده و چرا این اقدام را انجام می‌دهیم؟
	
	\subsection*{پاسخ}
	\lr{projected\_data} یا داده‌های پروژه‌شده، نمایش داده‌ها در فضای ویژگی کاهش‌یافته است که توسط مؤلفه‌های اصلی تعریف می‌شود. این داده‌ها با پروژه‌سازی داده‌های اصلی روی بردارهای ویژه انتخاب‌شده به‌دست می‌آیند:
	\[
	Y = X W
	\]
	که در آن:
	\begin{itemize}
		\item $X$ ماتریس داده‌های مرکزسازی‌شده با ابعاد $n \times d$ است.
		\item $W$ ماتریس بردارهای ویژه انتخاب‌شده با ابعاد $d \times m$ است.
		\item $Y$ ماتریس داده‌های پروژه‌شده با ابعاد $n \times m$ است.
	\end{itemize}
	
	هدف از این پروژه‌سازی، کاهش ابعاد داده‌ها با حفظ بیشترین واریانس ممکن است. این کار باعث ساده‌سازی تحلیل داده‌ها، کاهش نویز و بهبود عملکرد الگوریتم‌های یادگیری ماشین می‌شود.
	
\end{document}
