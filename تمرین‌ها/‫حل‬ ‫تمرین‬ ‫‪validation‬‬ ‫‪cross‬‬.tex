\documentclass[12pt]{article}

\usepackage{amssymb}
\usepackage{amsmath}
\usepackage{graphicx}
\usepackage{xepersian}
\settextfont{Yas}
\setdigitfont{A Iranian Sans}

\title{تمرین Cross-Validation}
\author{محمد صالح علی اکبری}
\date{تاریخ تحویل: ۲ خرداد}

\begin{document}
	
	\maketitle
	
	\section*{سؤال ۱}
	جمع‌بندی و نکات مهم در انتخاب روش \lr{Cross-Validation} را تشریح کنید.
	
	\subsection*{پاسخ}
	انتخاب روش \lr{Cross-Validation} بستگی به ویژگی‌های داده و ماهیت مسئله دارد. نکات مهم در زیر فهرست شده‌اند: 
	\begin{itemize}
		\item \textbf{K-Fold CV}: روش عمومی برای مسایل رگرسیون و طبقه‌بندی است که داده را به \(K\) بخش مساوی تقسیم می‌کند و در هر دور یک بخش برای آزمون و باقی برای آموزش استفاده می‌شود. این روش واریانس برآورد خطا را کاهش می‌دهد اما با افزایش \(K\) هزینه محاسباتی بیشتر می‌شود. :contentReference[oaicite:0]{index=0}
		\item \textbf{Leave-One-Out CV (LOOCV)}: حالت خاص K-Fold که در آن \(K\) برابر تعداد نمونه‌هاست. در هر مرحله یک نمونه برای آزمون و مابقی برای آموزش استفاده می‌شود. مناسب برای داده‌های بسیار کوچک اما هزینه محاسباتی بالایی دارد. :contentReference[oaicite:1]{index=1}
		\item \textbf{Stratified K-Fold CV}: نسخه‌ای از K-Fold برای مسائل طبقه‌بندی با توزیع نامتوازن کلاس‌ها که نسبت نمونه‌های هر کلاس در هر فولد حفظ می‌شود. موجب پایداری بیشتر در ارزیابی دقت می‌شود. :contentReference[oaicite:2]{index=2}
		\item \textbf{Time Series Split}: ویژه سری‌های زمانی است که در آن ترتیب داده‌ها باید حفظ شود و از داده‌های آینده در آموزش استفاده نمی‌شود. در هر مرحله از آخرین نقاط گذشته برای آموزش و دوره‌ای از داده‌های بعدی برای آزمون بهره می‌برد. :contentReference[oaicite:3]{index=3}
		\item \textbf{Group K-Fold}: برای داده‌هایی با ساختار گروهی (مانند مطالعات پزشکی) که تضمین می‌کند نمونه‌های یک گروه همگی در آموزش یا آزمون قرار گیرند تا از نشت اطلاعات جلوگیری شود. :contentReference[oaicite:4]{index=4}
	\end{itemize}
	برای مقایسه روش‌ها معمولاً از معیارهایی مانند \lr{RMSE}، \lr{Accuracy Score} و نظایر آن استفاده می‌شود و انتخاب عدد مناسب \(K\) بر اساس توازن بین دقت تخمین و هزینه‌ی محاسباتی صورت می‌گیرد. 
	
	\section*{سؤال ۲}
	فقط روش‌های عمومی (برای رگرسیون و طبقه‌بندی) را نام برده و دو روش با انتخاب خودتان را تشریح کنید.
	
	\subsection*{پاسخ}
	\paragraph{فهرست روش‌های عمومی:}
	\begin{itemize}
		\item Holdout (یا Hold-Out Validation)
		\item K-Fold Cross-Validation
		\item Leave-One-Out CV (LOOCV)
		\item Stratified K-Fold CV
		\item Repeated K-Fold CV
		\item Shuffle-Split CV
		\item Time Series Split
		\item Group K-Fold CV
	\end{itemize}
	
	\paragraph{1. K-Fold Cross-Validation:}
	در این روش، مجموعه‌داده به \(K\) بخش (فولد) مساوی تقسیم می‌شود. در هر تکرار، یک فولد به عنوان داده آزمون و \(K-1\) فولد باقی‌مانده به عنوان داده آموزش استفاده می‌شوند. این فرایند \(K\) بار تکرار شده و نتایج ارزیابی میانگین‌گیری می‌شود. 
	\begin{itemize}
		\item مزایا: استفاده بهینه از داده (تمام نمونه‌ها هم در آموزش و هم در آزمون مشارکت دارند)، کاهش واریانس در برآورد خطا.
		\item معایب: هزینه محاسباتی افزایش می‌یابد، مخصوصاً برای \(K\) بزرگ.
		\item موارد کاربرد: مسائل رگرسیونی و طبقه‌بندی عمومی که وابستگی زمانی یا گروه‌بندی داده وجود ندارد.
	\end{itemize}
	
	\paragraph{2. Stratified K-Fold CV:}
	نسخه بهبود‌یافته‌ای از \lr{K-Fold} برای مسائل طبقه‌بندی با داده‌های نامتوازن است. در این روش، نسبت نمونه‌های هر کلاس در هر فولد مشابه نسبت کل داده حفظ می‌شود.
	\begin{itemize}
		\item مزایا: جلوگیری از ایجاد فولدهای با پراکندگی نامناسب کلاس‌ها، بهبود ثبات و دقت ارزیابی در مسائل طبقه‌بندی.
		\item معایب: پیچیدگی اندکی بیشتر در پیاده‌سازی، فقط برای مسائل طبقه‌بندی کاربرد دارد.
		\item موارد کاربرد: طبقه‌بندی دودویی یا چندکلاسه با توزیع نامتوازن کلاس‌ها.
	\end{itemize}
	
\end{document}
